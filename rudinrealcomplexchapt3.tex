  \documentclass[12pt]{article}

\usepackage{amsmath, amsfonts,listings, amsthm,mathtools,graphicx}

\title{Rudin Real and Complex Analysis Chapter 3}
\author{Joseph Willard}
%\date{December,8,2015}

\newtheorem{theorem}{Theorem}[section]


\begin{document}
\maketitle

\section*{Problem 11}
To begin take the lebesgue integral of $fg \geq 1$, $\int_{\Omega}fgd\mu
\geq \int_{\Omega}d\mu$.
\begin{center}
$\int_{\Omega}fg d\mu \geq \int_{\Omega}d\mu$
\end{center}
Looking at the right side of the inequality,
\begin{align*}
\int_{\Omega}d\mu &= \sup \int_{\Omega}s d\mu\\  
&= \sup \sum_{i = 1}^{n}\alpha_{i}\mu (A_{i} \cap \Omega)\\
&= \mu(\Omega) = 1
\end{align*}
In the above $\alpha_{i} = 1$ and $A_{i}$ is all values from $\Omega$
that give the same $\alpha_{i}$. From this since
$\int_{\Omega}f(x)d\mu = \int_{\Omega}d\mu$ $f(x)=1$ which is just a
horizontal line thus $\mu (A_{i} \cap \Omega) = \mu(\Omega)$. Also
note that since $f(x)$ only produces one value there is only one
$\alpha_{i}$. From above we have shown that clearly
$\int_{\Omega}fgd\mu \geq 1$. Next since it is given that $f$ and $g$
are positive measurable functions we can utilize hoelders
inequality. First we begin with $p=1$ and $q=\infty$ and then with $p=\infty$ and $q = 1$.
\begin{align*}
1 &\leq \int_{\Omega}fg d\mu\\
  & \leq \Big \{ \int_{\Omega}f^{p}d\mu \Big \}^{\frac{1}{p}}\Big \{ \int_{\Omega}g^{q}d\mu \Big \}^{\frac{1}{q}}\\
  &\leq \int_{\Omega}f d\mu\\
\end{align*}
\begin{align*}
  1 &\leq \int_{\Omega}fg d\mu\\
  & \leq \Big \{ \int_{\Omega}f^{p}d\mu \Big \}^{\frac{1}{p}}\Big \{ \int_{\Omega}g^{q}d\mu \Big \}^{\frac{1}{q}}\\
  &\leq \int_{\Omega}g d\mu
\end{align*}
Since $\int_{\Omega}g d\mu \geq 1$ and $\int_{\Omega}f d\mu \geq 1$ it
is clear that $\int_{\Omega}f d\mu \cdot \int_{\Omega}g d\mu \geq 1$.


\section*{Problem 24}
\subsection*{(a)}
Break this up into 2 cases. The first case is $f < g$
\begin{align*}
\int ||f|^{p}-|g|^{p}|d\mu &\leq
\end{align*}

\section*{Problem 25}
\subsection*{$\int_{E}(\log(f))d\mu \leq \mu(E)\log(\frac{1}{\mu(E)})$}
To begin note $\int_{E}fd\mu \leq \int_{X}fd\mu = 1$ in the case $E
\subset X$, this comes directly from definition and more so from the
fact that $\mu$ and $f$ are both positive functions. To see this I am
looking at definition $1.23$,
\begin{center}
$\int_{X}fd\mu = \sup \sum_{i=1}^{n}\alpha_{i}\mu (A_{i}\cap X)$
\end{center}
Where $A_{i}= \{x : f(x) = \alpha_{i}\}$ so clearly
$\alpha_{i}=f(x)$. Now if we consider any subset $E \subset X$ coupled
with the fact that $f$ and $\mu$ are positive as stated above and in
the origninal question it is obvious that the sum produced from
$\int_{X}fd\mu$ would be bigger. Next consider the right side of the
inequality, we can rewrite this,
\begin{align*}
\mu(E)\log\bigg(\frac{1}{\mu(E)}\bigg) &=\mu(E)(\log(1)-\log(\mu(E)))\\
&=-\mu(E)\log(\mu(E))
\end{align*}
But what can we say about $\mu(E)$? 
\begin{align*}
\mu(E) &\leq \mu(X)\\ 
&= \mu(\bigcup_{i=1}^{n}E_{i})\\
\end{align*}
So arbitrarily I am just breaking $X$ into disjoint sets of size $E$ so one of
the $E_{i}$'s should actually be $E$ and by using the definition of a
measure this holds. But,

\begin{align*}
\mu(X) &= \mu(E_{1}\cap X) + \mu(E_{2}\cap X) + \cdots + \mu(E_{n}\cap X)\\
&< \alpha_{1} \mu(E_{1}\cap X) +\alpha_{2} \mu(E_{2}\cap X) + \cdots +\alpha_{n} \mu(E_{n}\cap X)\\
&\leq \int_{X}fd\mu = 1
\end{align*}
This implies $0< \mu(E) < 1$, since $\alpha_{i} > 0$. Utilizing what
we've just shown it is clear that $\log(\mu(E)) < 0$ since $\log(x) <
0$ for $0<x<1$ and further more we can infer that $-\mu(E)\log(\mu(E))
> 0$. Keeping all this in mind we can take our inequality and apply
Jensen's inequality (theorem $3.3$) and show,
\begin{align*}
\int_{E}(\log(f))d\mu &\leq \mu(E)\log(\frac{1}{\mu(E)})\\
e^{\int_{E}(\log(f))d\mu} &\leq e^{\mu(E)\log(\frac{1}{\mu(E)})}\\
\int_{E}e^{(\log(f))d\mu} &\leq e^{\mu(E)\log(\frac{1}{\mu(E)})}\\
\int_{E} f d\mu &\leq e^{\mu(E)\log(\frac{1}{\mu(E)})}\\
\end{align*}   
To finish this by definition $e^{x} > 1$ for $x > 0$ and what I
mentioned in the first sentence all the above inequalities hold.
\section*{Problem 26}
To begin let's look at $f(x)$. Since it is given that $f(x)$ is a
positive measurable function its image is a
subset or equal to $[0, \infty]$. Next we need to consider
$\log(f(x))$ in particular this function is negative when
$0<f(x)<1$. To approach this problem I am considering two scenarios,
first the case when $0<f(x)<1$ and second the case when $f(x) >
1$. Note that when $f(x) = 0$ or $f(x) = 1$ then both integrals are
equal (i.e either $-\infty$ or $0$). So in the first case when
$0<f(x)<1$ I will use hoelder's inequality first with $p = 1$ and $q =
\infty$, then vice-versa.

\begin{align*}
  \int_{0}^{1}f(x)\log(f(x))dx &\leq \int_{0}^{1}f(x)(-\log(f(x)))dx\\
  &\leq \Big\{ \int_{0}^{1}f(x)^{p}dx \Big \}^{\frac{1}{p}} \Big \{  \int_{0}^{1}(-\log(f(x)))^{q}dx \Big \}^{\frac{1}{q}}\\
  &\leq \Big\{ \int_{0}^{1}f(x)^{1}dx \Big \}^{\frac{1}{1}} \lim_{q
    \to\infty}\Big \{  \int_{0}^{1}(-\log(f(x)))^{q}dx \Big \}^{\frac{1}{q}}\\
  %&\leq \Big\{ \int_{0}^{1}f(x)dx \Big \} \Big \{  \int_{0}^{1}\infty dx \Big \}^{0}\\
  &\leq \int_{0}^{1}f(x)dx
\end{align*}
In the first inequality I write $-\log(f(x))$ to
obtain a function that has a positive image so I can use Hoelder's
inequality. I obtain my final inequality by noting that $\lim_{q
  \to\infty}\frac{1}{q} = 0$. Now with $p = \infty$ and $q = 1$,

\begin{align*}
  \int_{0}^{1}f(x)\log(f(x))dx &\leq \int_{0}^{1}f(x)(-\log(f(x)))dx\\
  &\leq \Big\{ \int_{0}^{1}f(x)^{p}dx \Big \}^{\frac{1}{p}} \Big \{  \int_{0}^{1}(-\log(f(x)))^{q}dx \Big \}^{\frac{1}{q}}\\
  &\leq \lim_{p\to\infty}\Big\{ \int_{0}^{1}f(x)^{p}dx \Big \}^{\frac{1}{p}} \Big \{  \int_{0}^{1}(-\log(f(x)))^{1}dx \Big \}^{\frac{1}{1}}\\
  %&\leq \Big \{  \int_{0}^{1}\infty dx \Big \}^{0}\Big \{  \int_{0}^{1}-\log(f(x))dx \Big \}\\
  &\leq \int_{0}^{1}-\log(f(x))dx
\end{align*}
From this we get the inequalitiess,
\begin{align*}
  \int_{0}^{1}f(x)(-\log(f(x)))dx &\leq \int_{0}^{1}f(x)dx\int_{0}^{1}-\log(f(x))dx\\
  -\int_{0}^{1}f(x)\log(f(x))dx &\leq -\int_{0}^{1}f(x)dx\int_{0}^{1}\log(f(x))dx\\
  \int_{0}^{1}f(x)\log(f(x))dx &\geq \int_{0}^{1}f(x)dx\int_{0}^{1}\log(f(x))dx\\
\end{align*}
In the case when $0<f(x)<1$. When dealing with $f(x) > 0$ we can use hoelder's inequaltiy 
without making adjustments, again starting with $p = 1$ and $q$
approaching infinity,
\begin{align*}
  \int_{0}^{1}f(x)\log(f(x))dx &\leq \Big\{ \int_{0}^{1}f(x)^{p}dx \Big \}^{\frac{1}{p}} \Big \{  \int_{0}^{1}\log(f(x))^{q}dx \Big \}^{\frac{1}{q}}\\
  &\leq \int_{0}^{1}f(x)dx
\end{align*}
Now with $p = \infty$, $q = 1$,

\begin{align*}
  \int_{0}^{1}f(x)\log(f(x))dx &\leq \Big\{ \int_{0}^{1}f(x)^{p}dx \Big \}^{\frac{1}{p}} \Big \{  \int_{0}^{1}\log(f(x))^{q}dx \Big \}^{\frac{1}{q}}\\
  &\leq \int_{0}^{1}\log(f(x))dx
\end{align*}
Of course we can change our term of integration from $x$ to the
respective $t$ and $s$ since they are only place holders which gives
us our result, $ \int_{0}^{1}f(x)\log(f(x))dx \leq \int_{0}^{1}f(s)ds \int_{0}^{1}\log(f(t))dt$.

\subsection*{Thoughts}
\begin{itemize}
\item It doesn't account for a positive measurable function that
  outputs values both less than and greater than $1$.
\end{itemize}

\end{document}